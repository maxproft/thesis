\documentclass[a4paper,12pt]{report}
\usepackage[utf8]{inputenc}
\usepackage{times}
\usepackage{graphicx}
\usepackage{url}
\usepackage{amsmath}
\usepackage{array}
\usepackage{mathtools}
\usepackage{ifthen}
\usepackage{mathrsfs}

\oddsidemargin 0.1in		% Left margin is 1in + this value
\textwidth 5.75in		% Right margin is not set explicitly
\topmargin 0in			% Top margin is 1in + this value
\textheight 8.5in			% Bottom margin is not set explicitly
%\setlength{\columnsep}{2cm}


\iffalse
These words need to be added to the dictionary:
Ginzburg, soliton, Runge, Kutta, CGLE, Fourier, Landau
\fi


\iffalse
\begin{figure}[h]
\centering
\includegraphics[width=4.6in]{correlation}
\caption{A plot of the correlation in the arrival time of photons for a thermal source. Photons are more likely to be detected shortly after another photon arrives. The axes have arbitrary units as the correlation will vary with the temperature.}
\label{corr} 
\end{figure}
\fi


\iffalse
Before I submit, make sure I check that the following have updated themselves properly.
Spell Check
Contents Page
Image reference numbers
Table reference numbers
Citation numbers
\fi








\begin{document}
\begin{titlepage}
\centering
{\ \\ }
\vspace{3cm}
{\bf  \fontsize{1.35cm}{2.5cm}\selectfont Soliton Solutions of the \vspace{0.5cm} Complex Ginzburg \\\vspace{0.5cm}Landau Equation}\\
\vspace{2cm}
{\Huge By Max Proft}\\
{\Huge u5190335}
\end{titlepage}










\newcommand{\unnum}[2]{
\ifthenelse{\equal{#1}{chapter}}{\chapter*{#2}}{
\ifthenelse{\equal{#1}{section}}{\section*{#2}}{
\ifthenelse{\equal{#1}{subsection}}{\subsection*{#2}}
{\chapter*{ERROR: #2}}}}
\addcontentsline{toc}{#1}{#2} 
}%This is the exception if no case matches!!!
\tableofcontents{}
\iffalse
The order is:
Chapter
Section
Subsection
Subsubsection (not shown in contents)
\fi












\unnum{chapter}{Abstract}
Abstract goes here.

\unnum{chapter}{Acknowledgements}
Chuck in some very cliché acknowledgements here.












\chapter{Introduction}
\section{Things to include:}
dissipative/nonconservative\\
Where do we see this happening?\cite{Ref2}\\
nonintegrable, limitations of calculating by hand, the reason for computers\\
why searching for solutions with machine learning is needed\\
\cite[pp.~215]{RefExample} 

\section{section}
Lorem ipsum dolor sit amet, consectetuer adipiscing elit.  
Etiam lobortis facilisissem.  
\subsection{subsection}
Lorem ipsum dolor sit amet, consectetuer adipiscing elit.  
Etiam lobortis facilisissem.  
\subsection{subsection 2}
Lorem ipsum dolor sit amet, consectetuer adipiscing elit.  
Etiam lobortis facilisissem.  

\section{Second Section}
 
Lorem ipsum dolor sit amet, consectetuer adipiscing elit.  
Etiam lobortis facilisissem.  













\chapter{Derivations of the CGLE}
\section{For a Quantum System}
\section{For a Classical System}












\chapter{Types of solitons}
\subsection{Analytic Solutions}

\subsection{Other Solutions}
Chaotic, but spatially localised.











\chapter{Mathematical Theory Behind Solving the CGLE}
\section{A Comparison of Different Methods}
Decide whether to use non linear, non-linear, nonlinear
Finite difference, what was used initially to get it all going, is bad because you do $\delta \psi / \delta x$, so over a small time period, the $\delta \psi$ term can lose several digits of precision, not good. There are a number of other ways for this to be solved though. 
\\\\

For the nonlinear schrodinger equation, (author) compared various methods for solving this. 
Other methods are EXAMPLE, and a split step fourier method. The first is better for large, slowly changing features, the second is better for sharp peaks, and the third is better for both of these.
Given the similarity between the CGLE and NLSE, this means that this should also be a reasonably good method to do this part as well. 

\section{Split Step Fourier and Runge-Kutta Method}
\subsection{Linear evolution (Fourier)}
Here we are solving the equation: 
$$\frac{d\psi}{dt}=A\frac{d^2\psi}{dx^2}+B\psi$$
Defining the fourier transform by: 
$$\hat \psi(f)=\int\limits_{-\infty}^\infty \psi(x)e^{-i2\pi x f} dx$$
By taking the fourier transform of both sides, the equation becomes:\\
$$\frac{d \hat \psi}{dt}= C(-2\pi i f)^2 D\hat \psi (f)+\hat \psi(f)= (C-D4 \pi^2 f^2) \hat \psi(f)$$
$$\implies \hat \psi_t(f) = e^{(C-D4\pi^2 f^2)t}\hat \psi_0(f)$$
And so upon inverse fourier transforming, we get the solution to this DE, where $\hat \psi_t(f)$ is the fourier transform at time $t$.
The Fourier transform can be done numerically with a discrete fourier transform, as described in a later subsection.

\subsection{Nonlinear evolution (Runge-Kutta)}
Here we are solving the equation:
$$\frac{d\psi}{dt}=C|\psi|^2\psi + D|\psi|^4\psi$$
The runga kutta method allows you to numerically solve a 1st order differential equation. which are of the form $\frac{dy}{dt}=f(t,y)$. Similar to Euler's method (which is basically just doing a derivative from first principles), except that the runge kutta method has better convergence than this. (it goes as $h^5$ as opposed to Euler's Method which is $h^2$, the backward euler formula, $h^2$, or improved euler formula which goes as $h^3$. Here $h$ is the step size. 
\\\\
The steps for doing this are as follows:
Step 1: define $f(t,y)=f(y)=C|y|^2 y  +D|y|^4 y$
Step 2: Have an initial value for $t$ and $y$, and step size and number of steps
Step 3: define the following:
$$K1 =f(t,y)=f(y)$$
$$K2 =f(t+0.5h,y+0.5*h*K1)=f(y+h*K1/2)$$
$$K3 =f(t+0.5h,y+0.5*h*K2)=f(y+h*K2/2)$$
$$K4 =f(t+h,y+h*K3)=f(y+h*K3)$$
Step 4: we then progress forwards $t$ and $y$ in the following way:
$$t\rightarrow t+h$$
$$y\rightarrow y+(K1+2K2+2K3+K4)*h/6$$


\subsection{The relationship between a Fourier Transform and a Discrete Fourier Transform}
\subsection{The relationship between a Fourier Transform and a Discrete Fourier Transform}
I am pretty sure this entire section needs to be rewritten.\\
We can approximate any function with a Fourier series. (by applying periodic boundary conditions)\\
$$f(t)=\sum\limits_{-\infty}^{\infty} c_n e^{i n t}$$
Multiplying both sides by $e^{-i m t}$ and integrating from 0 to $2\pi$, and dividing by $2\pi$:
$$\frac{1}{2\pi} \int\limits_{0}^{2\pi} f(t)e^{-i m t}dt = \frac{1}{2\pi} \sum \limits_{n=-\infty}^{\infty} c_n \int\limits_{0}^{2\pi} d^{i (n-m) t} dt= c_m$$
$$\frac{1}{2\pi} \int\limits_{0}^{2\pi} f(t)e^{-i m t}dt =c_m$$
We use the substitution:
$$t = \frac{2\pi}{L}  x \implies x = \frac{L}{2\pi} t$$
$$dt = \frac{2\pi}{L}  dx$$
And then this becomes:
$$c_m = \frac{1}{L} \int\limits_{0}^{L} f(x) e^{-i m 2\pi x/L} dx $$

For a fourier series $\{c_n\}$, the fourier transform is given given by:

$$\hat f (freq) = \sum\limits_{-\infty}^{\infty} c_n \mathscr{\hat F}(e^{i n 2 \pi x/L})(freq)$$ 

But we have $$\mathscr{\hat F}(e^{i n 2 \pi x/L}) = \int\limits_{-\infty}^\infty e^{i n 2 \pi x/L}e^{-i 2\pi freq x} dx $$
$$= \int\limits_{-\infty}^\infty e^{i x (n 2 \pi /L - 2\pi freq )}dx = 2\pi \delta (n 2\pi/L -2\pi freq)= \delta (n /L - freq)$$

And so we get:
$$\hat f (freq) = \sum\limits_{-\infty}^{\infty} c_n \delta(freq.- n/L)$$
$$= \sum\limits_{-\infty}^{\infty} c_n /L \delta(freq. L- n)$$

If we inverse fourier transform we get the following.
We can also use $x/L=n/N$
$$Data = spacial,  Fata=frequency$$


I don’t think the following is useful:
$$f(x)= \sum\limits_{n=-\infty}^\infty f_n e^{2\pi i n x/L} $$
If we multiply both sides by $e^{-2\pi i m x/L}$ and integrate we get:
$$\int \limits_{-\infty}^\infty f(x)e^{2\pi i n x/L} dx = \sum\limits_{n=-\infty}^\infty  f_n  \int \limits_{-\infty}^\infty  e^{i 2 \pi x/L (n-m)} dx  =  \sum\limits_{n=-\infty}^\infty  f_n  \int \limits_{-\infty}^\infty  e^{i 2 \pi x (n-m)/L} dx  $$
$$=  \sum\limits_{n=-\infty}^\infty  f_n \delta((n-m)/L) = f_n \delta((n-m)/L)$$





\section{Testing the Code}
\subsection{Comparison with Analytic Solutions}
Find exact solutions (to both CGLE and NLSE), and show that it matches well.
After $x$ amount of time, how accurate is the answer
\subsection{Decreasing the step size and Determining Appropriate Values}
Find interesting solitons and show that we can reproduce these solutions.\\\\
For unusual, and non-analytic solutions, especially spiny solitons, as you decrease the step size, after a length of time $t$, how close is the answer to the limit? I.e. make the time step size 0.1, and propagate forwards in time until t=10. repeat with step sizes 0.01, 0.001 and plot the final answer. It should approach a constant value. Take the step size to be the smallest value such that the solution is within 1\% of the answer with the previous step size.















\chapter{Machine Learning Theory}
I want to enter in an image, and get it to tell me if it is oscillating/etc.\\
I want another machine learning algorithm to predict the required precision, and wait time before recording the output.\\
Monte Carlo and gradient descent to find new solns.\\
\section{Different Options for the Machine Learning Algorithm}
\section{A detailed look at neural networks}
\section{Training and Testing the Algorithm}











\chapter{The result from much work}
\section{The sorts of things that I found}












\unnum{chapter}{Conclusion}
\unnum{chapter}{Appendix A}
\unnum{chapter}{Appendix B}

\begin{thebibliography}{9}
%Introduction
\bibitem{RefExample}
    Name \emph{Title},
    Publishing info, etc
\bibitem{Ref2}
    Another Guy \emph{title2}
    pub,etc.
%Derivation of CGLE

%Types of Solitons

%Mathematical theory behind solving CGLE

%Machine Learning

%My solitons


\end{thebibliography}


\end{document}