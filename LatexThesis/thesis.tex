\documentclass[a4paper,12pt]{report}
\usepackage[utf8]{inputenc}
\usepackage{times}
\usepackage{graphicx}
\usepackage{url}
\usepackage{amsmath}
\usepackage{array}
\usepackage{mathtools}
\usepackage{ifthen}

\oddsidemargin 0.1in		% Left margin is 1in + this value
\textwidth 5.75in		% Right margin is not set explicitly
\topmargin 0in			% Top margin is 1in + this value
\textheight 8.5in			% Bottom margin is not set explicitly
%\setlength{\columnsep}{2cm}


\iffalse
These words need to be added to the dictionary:
Ginzburg, soliton, Runge, Kutta, CGLE
\fi


\iffalse

\begin{figure}[h]
\centering
\includegraphics[width=4.6in]{correlation}
\caption{A plot of the correlation in the arrival time of photons for a thermal source. Photons are more likely to be detected shortly after another photon arrives. The axes have arbitrary units as the correlation will vary with the temperature.}
\label{corr} 
\end{figure}

\fi








\begin{document}
\begin{titlepage}
\centering
{\ \\ }
\vspace{3cm}
{\bf  \fontsize{1.35cm}{2.5cm}\selectfont Soliton Solutions of the \vspace{0.5cm} Complex Ginzburg \\\vspace{0.5cm}Landau Equation}\\
\vspace{2cm}
{\Huge By Max Proft}\\
{\Huge u5190335}
\end{titlepage}










\newcommand{\unnum}[2]{
\ifthenelse{\equal{#1}{chapter}}{\chapter*{#2}}{
\ifthenelse{\equal{#1}{section}}{\section*{#2}}{
\ifthenelse{\equal{#1}{subsection}}{\subsection*{#2}}
{\chapter*{ERROR: #2}}}}
\addcontentsline{toc}{#1}{#2} 
}%This is the exception if no case matches!!!
\tableofcontents{}
\iffalse
The order is:
Chapter
Section
Subsection
Subsubsection (not shown in contents)
\fi












\unnum{chapter}{Abstract}
Abstract goes here.

\unnum{chapter}{Acknowledgements}
Chuck in some very cliché acknowledgements here.












\chapter{Introduction}
\section{Things to include:}
dissipative/nonconservative\\
Where do we see this happening?\cite{Ref2}\\
nonintegrable, limitations of calculating by hand, the reason for computers\\
why searching for solutions with machine learning is needed\\
\cite[pp.~215]{RefExample} 

\section{section}
Lorem ipsum dolor sit amet, consectetuer adipiscing elit.  
Etiam lobortis facilisissem.  
\subsection{subsection}
Lorem ipsum dolor sit amet, consectetuer adipiscing elit.  
Etiam lobortis facilisissem.  
\subsection{subsection 2}
Lorem ipsum dolor sit amet, consectetuer adipiscing elit.  
Etiam lobortis facilisissem.  

\section{Second Section}
 
Lorem ipsum dolor sit amet, consectetuer adipiscing elit.  
Etiam lobortis facilisissem.  













\chapter{Derivations of the CGLE}
\section{For a Quantum System}
\section{For a Classical System}












\chapter{Types of solitons}













\chapter{Mathematical Theory Behind Solving the CGLE}
Decide whether to use non linear, non-linear, nonlinear
\section{A Comparison of Different Methods}
\section{Split Step Fourier and Runge-Kutta Method}
\subsection{Linear evolution (Fourier)}
\subsection{Nonlinear evolution (Runge-Kutta)}
\subsection{The relationship between a Fourier Transform and a Discrete Fourier Transform}
















\chapter{Machine Learning Theory}
I want to enter in an image, and get it to tell me if it is oscillating/etc.\\
I want another machine learning algorithm to predict the required precision, and wait time before recording the output.\\
Monte Carlo and gradient descent to find new solns.\\
\section{Different Options for the Machine Learning Algorithm}
\section{A detailed look at neural networks}
\section{Training and Testing the Algorithm}











\chapter{The result from much work}
\section{The sorts of things that I found}












\unnum{chapter}{Conclusion}
\unnum{chapter}{Appendix A}
\unnum{chapter}{Appendix B}

\begin{thebibliography}{9}
%Introduction
\bibitem{RefExample}
    Name \emph{Title},
    Publishing info, etc
\bibitem{Ref2}
    Another Guy \emph{title2}
    pub,etc.
%Derivation of CGLE

%Types of Solitons

%Mathematical theory behind solving CGLE

%Machine Learning

%My solitons


\end{thebibliography}


\end{document}